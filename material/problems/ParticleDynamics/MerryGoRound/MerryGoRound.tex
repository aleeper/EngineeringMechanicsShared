% File: MerryGoRound.tex
% Author: Adam Leeper
%------------------------------------------------------------------------------
\providecommand{\isolatedBuild}[1]{#1}% Fallback definition to build normally.
\isolatedBuild{
  \documentclass[11pt,letterpaper]{book}
  %\documentclass[11pt,letterpaper]{book}

% aleeper: I think these are needed for Paul's macros?
\usepackage{epsfig}
\usepackage{epstopdf}

%\makeatletter
%\typeout{The import path is \import@path}
%\makeatother

\usepackage{import}

\subimport{./}{packagesMitiguy.sty}
\subimport{./}{macrosMitiguy.tex}
\subimport{./}{PageStylesMitiguy.tex}
\subimport{./}{macrosLeeper.tex}
   % Found via TEXINPUTS environment variable.
  \isolatedBuildHeader{2D Angular Momentum for a Child on a Merry-Go-Round}
                      {2D Angular Momentum for a Child on a Merry-Go-Round}
}
%%%
%%%
%%%
A child, modeled as a particle $Q$, of mass $m^Q = 50$ kg stands at the
center of a merry-go-round $B$ of radius $R = 2$ m and mass $m^B = 200$ kg.
Let $\uvecz{b}$ be a unit vector vertically upward (perpendicular to the
surface of $B$). Let the center of the merry-go-round be coincident with a
point $N_o$ which is fixed in a Newtonian reference frame $N$. The angular
velocity of $B$ in $N$ can be described as function of time as
$\angvel{B}{N} = \omega_z(t)~\uvecz{b}$.

Initially, $\omega_z( t = 0 ) = 6$ rad/s.
%The child then becomes dizzy and begins to walk in an outward spiral, such that his position in meters (expressed as a function of time $t$ in seconds) is given by:
%\\[0.25pc]
%$$\posvec{N_o}{\cm{C}} = \frac{t}{4}\cos(t)~\uvecx{b} + \frac{t}{4} \sin(t)~\uvecy{b}$$
%
The child then begins to walk outward such that his position in meters (expressed as a function of time $t$ in seconds) is given by:
$\posvec{N_o}{Q} = t~\uvecx{b}$.

Neglecting sources of friction and air-resistance,
we can assume there are \textbf{no external moments} about \uvecz{b}
on the system consisting of $B$ and $Q$.

\textbf{Hint:} It is important in this problem for you to distinguish between
$\omega_z(t=0)$, which is a single number, and the variable $\omega_z(t)$.
Don't combine them together or cancel them out.

\begin{enumerate}
  \item Calculate the angular momentum of the child about $N_o$ at time $t$
    in terms of $m^Q$, $t$, and $\omega_z(t)$.
  \item Using part (a) and the information about the initial state of the
    system, find an expression for $\omega_z(t)$.
  \item Hence, calculate $\omega_z(t=2)$.
  %\item In reality the child is not a particle, but has some ``extent"
  %  horizontally. Modeling the child as a cylinder of radius $r = 0.2$ m and
  %  height $h = 1$ m, recalculate $\omega_z(t=2)$ and determine the
  %  difference from part (b) as a percentage. In your judgement, is the
  %  particle model reasonable?
\end{enumerate}
%
\isolatedBuildFooter
