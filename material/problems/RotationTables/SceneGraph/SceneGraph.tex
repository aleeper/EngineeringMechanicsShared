%   File: SceneGraph.tex
% Author: Adam Leeper
%------------------------------------------------------------------------------
%\\[0.45pc]
\providecommand{\isolatedBuild}[1]{#1}% fallback definition lets this file build normally
\isolatedBuild{
  \documentclass[11pt,letterpaper]{book}
  %\documentclass[11pt,letterpaper]{book}

% aleeper: I think these are needed for Paul's macros?
\usepackage{epsfig}
\usepackage{epstopdf}

%\makeatletter
%\typeout{The import path is \import@path}
%\makeatother

\usepackage{import}

\subimport{./}{packagesMitiguy.sty}
\subimport{./}{macrosMitiguy.tex}
\subimport{./}{PageStylesMitiguy.tex}
\subimport{./}{macrosLeeper.tex}
   % Must be found via TEXINPUTS environment variable.
  \isolatedBuildHeader{Rotation Matrices}
                      {Computer Graphics Scene Graph:
                        Working with Multiple Rotation Matrices}
}
%%%
%%%
%%%
In computer graphics and robotics, a \textit{scene graph} is a structure which
holds position and rotation information for a set of frames.
Typically this structure gives you rotation matrices relating each frame to a
single common ``world'' frame; it is then up to you to compute the quantities
you need for a particular use.
%
\\[0.45pc]
For this problem, assume you obtain from the scene graph the rotation matrices
\dircos{N}{A}, \dircos{N}{B}, \dircos{N}{C} which relate the right-handed,
orthogonal unit vectors of frames \basis{N}, \basis{A}, \basis{B}, and \basis{C}.
\\[-0.75pc]
\begin{enumerate}
  \item
    Given \dircos{N}{A}, explain how to obtain the inverse rotation
    matrix, \dircos{A}{N}.
    \\[0.45pc]
    \Solution {}{1.0\linewidth}{
      You \underline{transpose} \dircos{N}{A} to get \dircos{A}{N}.
      In other words, $\dircos{A}{N} = (\dircos{N}{A})^T$.
      \\[0.45pc]
      Note: In linear algebra, a matrix with the property
      $\dircos{N}{A} * \dircos{A}{N} \equals[\;] I$ is
      called an \textit{orthogonal} matrix.
    }
%
  \item
    Show how to use the above quantities to form the rotation
    matrix relating frame A and C.
    \\[1.0pc] $\dircos{A}{C} \equals[\;]$
      $\hidemath[0cm]{
        \dircos{A}{N} \mult[\;] \dircos{N}{C}
        \equals[\;] (\dircos{N}{A})^T * \dircos{N}{C}
        }$
%
\end{enumerate}
%
\isolatedBuildFooter
