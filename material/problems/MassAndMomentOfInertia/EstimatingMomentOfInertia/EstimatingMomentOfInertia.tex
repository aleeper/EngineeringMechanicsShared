% File: MomentOfInertia.tex
% Author: Adam Leeper
%------------------------------------------------------------------------------
\providecommand{\isolatedBuild}[1]{#1}% Fallback definition to build normally.
\isolatedBuild{
  \documentclass[11pt,letterpaper]{book}
  %\documentclass[11pt,letterpaper]{book}

% aleeper: I think these are needed for Paul's macros?
\usepackage{epsfig}
\usepackage{epstopdf}

%\makeatletter
%\typeout{The import path is \import@path}
%\makeatother

\usepackage{import}

\subimport{./}{packagesMitiguy.sty}
\subimport{./}{macrosMitiguy.tex}
\subimport{./}{PageStylesMitiguy.tex}
\subimport{./}{macrosLeeper.tex}
   % Found via TEXINPUTS environment variable.
  \isolatedBuildHeader{Estimating Inertia}
                      {Estimating Inertia}
}
%%%
%%%
%%%
Consider a man of mass 100 kg standing up with his arms at his sides. He is
\textit{approximately} 2 meters tall, 0.5 meters wide, and 0.25 meters deep.
%
\begin{enumerate}
  \item Sketch the system, with \uvecy{b} pointed from feet to head and
    \uvecz{b} pointed from back to front.
  \item Compute the man's \textbf{approximate} moments of inertia about his
    center of mass for \uvecxyz{b}, using the \textbf{rough model} that he is
    a box of uniform density. That is, compute \iscalarxx{B/\cm{B}},
    \iscalaryy{B/\cm{B}}, and \iscalarzz{B/\cm{B}}.
  \item From (a), compute the moments of inertia about a point $B_1$ located
    between his feet, for \uvecxyz{b}.
  \item Briefly describe what happens to these values when the man crouches
    and ``tucks" his arms and legs (e.g. doing a cannon-ball into a pool,
    or a backflip).
  %\item Comment on the accuracy of this model. Is it within 1\%? 50\%?
  %  Totally bogus?
  %\item Explain briefly how you might go about computing (or otherwise
  %  determining) a more accurate estimate of the moments of inertia for the man.
  %\item With reference to moment of inertia, explain why it is advantageous
  %  to ``tuck" when performing a backflip.
\end{enumerate}

%\ScoreOutOfPointsOnBottomRightOfPage{20}
%
\isolatedBuildFooter
