%   File: Golfer.tex
% Author: Adam Leeper
%------------------------------------------------------------------------------
%\\[0.45pc]
\providecommand{\isolatedBuild}[1]{#1}% Fallback definition to build normally.
\isolatedBuild{
  \documentclass[11pt,letterpaper]{book}
  %\documentclass[11pt,letterpaper]{book}

% aleeper: I think these are needed for Paul's macros?
\usepackage{epsfig}
\usepackage{epstopdf}

%\makeatletter
%\typeout{The import path is \import@path}
%\makeatother

\usepackage{import}

\subimport{./}{packagesMitiguy.sty}
\subimport{./}{macrosMitiguy.tex}
\subimport{./}{PageStylesMitiguy.tex}
\subimport{./}{macrosLeeper.tex}
   % Found via TEXINPUTS environment variable.
  \isolatedBuildHeader{Projectile Motion Examples}
                      {Projectile Motion of a Golf Ball
    %\footnote{Answer: With $v_o = 80 ft/s at \degrees{45}, d = 166 ft.
    %Adapted from exercise 12-102, Engineering Mechanics Dynamics,
    %R.C. Hibbeler, 12th ed., Pearson, 2010.}
    %\footnote{Answer: With $v_o = 20 ft/s at \degrees{60}, t = ??? sec.
    %Adapted from exercise 12-102, Engineering Mechanics Dynamics,
    %R.C. Hibbeler, 12th ed., Pearson, 2010.}
  }
}
%%%
%%%
%%%
\begin{minipage}{0.50\textwidth}
  A golf ball, $Q$, is struck with an initial velocity of 20 m/s in the
  direction shown by the arrow.
  Let the earth be a Newtonian frame, \basis{N}, and assume the gravitational
  constant is $g = 9.81$ m/s$^2$.
  Neglecting air resistance, we can model the forces on the ball as
  $\force{Q} \equals[] \minus[] mg~\uvecy{n}$.
\end{minipage}
\hfill
\begin{minipage}{0.43\textwidth}
  \includegraphicsAB[width=\linewidth]{golfer-solution.png}{golfer.png}
\end{minipage}
%
\\[0.45pc]
\begin{enumerate}
  \item Introduce appropriate measures to describe the kinematics of the ball.
    %
    \SolutionNoSpace{\\[0.45pc]}{0.98\linewidth}{
      The hardest part of this problem (indeed, most dynamics problems) is
      picking good identifiers. There is not a single ``right'' choice,
      but certain choices result in easier math at the end.
      %
      \\[0.45pc]
      Here, we introduce a basis \basis{B} with \uvecx{b} parallel to the
      slope and \uvecy{b} perpendicular to the slope. Then, we use \uvecxy{b}
      measures of the ball's position from \origin{N}.
      %
      \\[0.45pc]
      \begin{minipage}{0.5\linewidth}
        \center
        \rotationTable{n}{b}
          {\cos\theta_B}{-\sin\theta_B}{0}
          {\sin\theta_B}{\cos\theta_B}{0}
          {0} {0} {1}
        \\[1.0pc]
      \end{minipage}
      \begin{minipage}{0.5\linewidth}
        \center
        \begin{tabular}{lr@{\equals[\;]}l}
          Let
          & $\posvec{\origin{N}}{Q}$
          & $x~\uvecx{b} \plus y~\uvecy{b}$
          \\[0.45pc]
          Then
          & $\vel{Q}{N}$
          & $\xdot~\uvecx{b} \plus \ydot~\uvecy{b}$
          \\[0.45pc]
          & $\accel{Q}{N}$
          & $\xddot~\uvecx{b} \plus \yddot~\uvecy{b}$
        \end{tabular}
      \end{minipage}
    }
    %
  \item Use Newton's law to \textbf{show} how to get a pair of \textbf{scalar}
    differential equations describing the ball's motion.
    %
    \SolutionNoSpace{\\[0.45pc]}{0.98\linewidth}{
      \begin{minipage}{0.99\linewidth}
        Newton's law allows us to relate the ball's kinematics to the forces
        acting on the ball:
        \center
        \begin{tabular}{r@{\equals[\;]}l}
          $\force{Q}$
          & $m \accel{Q}{N}$
          \\[0.45pc]
          $ \minus[] mg~\uvecy{n}$
          & $m \xddot~\uvecx{b} \plus m \yddot~\uvecy{b}$
          \\[0.45pc]
        \end{tabular}
      \end{minipage}
      %
      \begin{minipage}{0.99\linewidth}
        We form scalar equations by dotting:
        \center
        \begin{tabular}{lr@{\equals[\;]}l@{\symmetricSpace{1cm}{$\longrightarrow$}}l@{\equals[\;]}l}
          dot with $\uvecx{b}$:
          & $\minus mg\sin\theta_B$
          & $m \xddot$
          & $\minus g \sin\theta_B$
          & $\xddot$
          \\[0.45pc]
          dot with $\uvecy{b}$:
          & $\minus mg\cos\theta_B$
          & $m \yddot$
          & $\minus g \cos\theta_B$
          & $\yddot$
        \end{tabular}
      \end{minipage}
    }
  \item Determine the \textbf{time} that the ball is in the air.
    %
    % $$ t \equals[\;] \hidemath{3.586 \; \mathrm{sec}} $$
    %
    \SolutionNoSpace{\\[0.45pc]}{0.98\linewidth}{
      Any 2nd-order differential equation of the form
      $\xddot = \text{constant}$
      can be solved using the familiar equations for
      ``constant-acceleration''.
      %
      $$s_f \minus s_i = v_i t \plus \frac{1}{2}at^2$$
      %
      The equations above for $\xddot$ and $\yddot$ both meet this requirement,
      so we can relate the initial conditions, final conditions, and time as:
      $$x_f \minus x_i = \xdot_i t \plus \frac{1}{2}\xddot t^2
        \symmetricSpace{1cm}{\text{and}}
        y_f \minus y_i = \ydot_i t \plus \frac{1}{2}\yddot t^2$$
      %
      We can solve for $\xdot_i$ and $\ydot_i$ using the initial condition:
      %
      $$(\vel{Q}{N})_i
        \equals (20~\mathrm{\frac{m}{s}})\cos\degrees{60}~\uvecx{b}
        \plus (20~\mathrm{\frac{m}{s}})\sin\degrees{60}~\uvecy{b}
        \equals \xdot_i~\uvecx{b} \plus \ydot_i~\uvecy{b}
      $$
      %
      Finally, we use the physical insight that the ball starts at
      $y_i \equals[] 0$, and will hit the ground again when $y_f \equals[] 0$.
      \center
      \begin{tabular}{r@{\equals[\;]}l}
        $y_f \minus y_i$
        & $\ydot_i t \plus \frac{1}{2}\yddot t^2$
        \\[0.45pc]
        $0 \minus 0$
        & $(10\sqrt{3} \mathrm{\frac{m}{s}})t
          \plus \frac{1}{2}(-9.81 \mathrm{\frac{m}{s^2}} \cos\degrees{10}) t^2$
        \\[0.45pc]
        0
        & $10 \sqrt{3} \mathrm{\frac{m}{s}}\minus
          \frac{1}{2} (9.81 \mathrm{\frac{m}{s^2}} \cos\degrees{10}) t$
        \\[0.45pc]
        $t$
        & $3.586 \; \mathrm{sec}$
      \end{tabular}
    }
\end{enumerate}
%
\isolatedBuildFooter
